
% ---
% Trabalhos Relacionados
% ---
\chapter{Trabalhos Relacionados}
\label{cap:trab_relac}
% ---



A literatura sobre modelos epidemiológicos é vasta, levando em conta aspectos transientes~\cite{ganesh2005effect} e estacionários~\cite{keeling2005networks, ratton}, bem como infecções endógenas e exógenas~\cite{zhang2017contact,zhang2015Network,zhang2018more}. Entretanto, não é de nosso conhecimento nenhum trabalho  que tenha analisado modelos analíticos levando em consideração atacantes estratégicos, de capacidade limitada, capazes de causar infecções exógenas, gerando expressões para a probabilidade de infecção dos nós.


Este trabalho é uma extensão de~\cite{rufino2018contaminaccao}.  Em~\cite{rufino2018contaminaccao} propusemos o modelo epidemiológico analisado no presente artigo.  Dentre as principais contribuições do presente trabalho, que não constavam em~\cite{rufino2018contaminaccao}, destacamos três: $(i)$ aproveitando-se de resultados recentes apresentados em~\cite{zhang2018more}, derivamos um método iterativo  para calcular a probabilidade de infecção dos nós.  Em seguida, apresentamos $(ii)$   fórmulas fechadas para aproximar a probabilidade de infecção assim como   $(iii)$   resultados de simulação. 

A proliferação de \emph{malware} e a formação de grandes \emph{botnets} permitem a execução de ataques distribuídos de negação de serviço com volume capaz de afetar serviços com grande capacidade \cite{kolias2017ddos, dynddos}. O crescimento da Internet das Coisas (IoT)~\cite{peterson}, combinado com as vulnerabilidades presentes nestes dispositivos e a dificuldade de atualizá-los criaram um ambiente propício para construção de \emph{botnets} \cite{angrishi2017turning}.  
As caracterizações e comportamentos observados em \emph{malware} real podem ser utilizados para parametrizar nossos modelo e simulador. Nossos modelos e o simulador são gerais o suficiente para serem  aplicados a novos \emph{malwares} que venham a ser identificados e caracterizados.   


Vários trabalhos na literatura estudam o comportamento e a evolução de \emph{malwares} \cite{antonakakis2017understanding, marzano2018evolution}.  Existe uma  constante evolução dos \emph{malwares} por parte dos atacantes.  \emph{Assim, a transição do estado infectado para o estado  suscetível  considerada neste trabalho pode refletir o fato de que um nó infectado, após aplicar uma contramedida, voltou a se tornar suscetível com relação a novas variantes de um mesmo \emph{malware}.} 





