
% resumo em português
\setlength{\absparsep}{18pt} % ajusta o espaçamento dos parágrafos do resumo
\begin{resumo}
Todos os dias sistemas são comprometidos por códigos maliciosos e  participam de campanhas de ataques a sistemas computacionais. 
Assim, os usuários enfrentam um dilema com relação a quais contramedidas tomar: duras (por exemplo, vacinação), suaves (por exemplo, reinicialização e rejuvenescimento) ou nenhuma contramedida. Para resolver esse dilema, uma opção é tomar proveito de  modelos analíticos. Neste trabalho, apresentamos uma proposta para carcterização do estado estacionário de modelos epidêmicos em que o atacante é estratégico e tem uma capacidade de ataque finita. Para tanto, é analisado os estados mais prováveis do modelo, indicando suas propriedades; apresenta-se fórmulas fechadas que  aproximam a probabilidade de infecção e contrastamos os \emph{insights} do modelo com simulações. Simulações suportam qualitativamente as observações do modelo epidêmico e estendem a análise permitindo distribuições gerais.

 \textbf{Palavras-chave}: epidemia, contaminação em redes, segurança da informação.
\end{resumo}