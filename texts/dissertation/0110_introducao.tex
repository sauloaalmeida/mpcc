
\chapter{Introdução}


	\textbf{Motivação} Epidemias em rede de computadores são onipresentes, todos os dias sistemas são comprometidos por códigos maliciosos que executam ações sem consentimento do seu dono legítimo (\textit{botnets}), os quais participam de atividades criminosas, como a espionagem industrial, sequestro de dados e transmissão de pornografia infantil, como também de atos de guerra, provocados por conflitos de estados ou grupos ideológicos, como ataques de negação de serviço (\textit{DoS}) e transmissão de informações falsas por meio de usuários legítimos. Tais tipos de ataques originam-se de centrais de comando e controle e fazem uso de dispositivos que são comprometidos a partir de infecções endógenas (i.e., entre vizinhos na rede local) ou exógenas (i.e., a partir de um dispositivo de uma rede remota); em todo caso, há um adversário (ou grupo) que intencionalmente faz as contaminações de seus códigos maliciosos (\textit{malwares}) para que possam exercer seu controle remotamente. 


    \textbf{Desafios} 
    Dentre os desafios enfrentados pelos administradores de sistemas, destacamos o dilema entre vacinar seus dispositivos (e.g., aplicando \emph{patches}) ou esperar e reiniciar certos processos, ou o sistema como um todo, de tempos em tempos (e.g., para fazer o rejuvenencimento do mesmo).  Embora a vacinação seja mais efetiva, ela pode envolver efeitos colaterais que indisponibilizem o sistema por um longo tempo e isso pode ser inviável, e.g., sistemas de controle industrial (ICS).  Para lidar com o \emph{tradeoff} entre aplicar contramedidas mais fortes ou  suaves e os possíveis custos associados a uma invasão, é fundamental ter um melhor entendimento sobre a probabilidade de infecção dos nós da rede.  Entretanto, ainda existem muitas questões em aberto no que concerne  a caracterização da probabilidade de infecção frente a atacantes estratégicos e riscos que os sistemas assumem por sua estratégia de aplicação de contramedidas.

    \textbf{Objetivo e metodologia} Neste trabalho, o objetivo é apresentar a proposta de tese de doutorado para desenvolvimento de modelos analíticos que sirvam de base para a análise de segurança em redes de computadores. Para cumprir tal objetivo, nesta qualificação é proposta a metodologia de: caracterização de comportamento dos nós em rede, em função da probabilidade de infecção, como em \cite{rufino2018contaminaccao}, da resposta resposta dos sistemas distribuídos em função dos ataques, como \cite{avritzer2019pptam}, e da medida de relevância de nós em função das métricas estabelecidas (centralidade), que é a proposta de integração dos trabalhos anteriores. Deste modo, a proposta é estabelecer formas de caracterização e estabelecimento de medidas que vão desde a infecção em sua origem, até os efeitos que esses ataques exercem sobre os sistemas computacionais em rede, que auxiliem a tomada de decisão e relevância do investimento em contramedidas eficazes de acordo com as ameaças e seus possíveis efeitos.
    

	\textbf{Lacunas no estado da arte} Existe uma ampla literatura sobre epidemias em redes de computadores, cuja base matemática remonta às epidemias biológicas.  Embora as epidemias em redes de computadores e biológicas tenham semelhanças entre si, elas também possuem importantes distinções. Dentre as distinções  destacamos o fato de que o atacante da rede de computadores pode ser estratégico, com alguma capacidade limitada, o qual pode varrer a rede completa na busca por nós vulneráveis. Modelos matemáticos levando em conta este tipo de comportamento são escassos, e não é de nosso conhecimento  pesquisa anterior que tenha derivado fórmulas fechadas para a probabilidade de infecção de nós neste cenário. Quando se trata de sistemas distribuídos, heterogêneos, tal como as nuvens, as informações disponíveis sobre os sistemas hospedados são superficiais, e neste cenário é possível coexistir ameaças e vulnerabilidades. Portanto a pergunta que desejamos responder é: poderia a configuração  (capacidades de processamento, armazenamento, comunicação, estratégia de contramedidas) fornecida por um provedor de nuvem ou centro de dados, auxiliar na segurança dos sistemas computacionais? E quais são as métricas necessárias, para a tomada de decisão?
	
	

	\textbf{Contribuições}:
	($i$) \textbf{Análise do modelo analítico de epidemias}  análise do  modelo analítico de epidemias proposto em~\cite{rufino2018contaminaccao}, indicando forma de parametrizá-lo e  analisando seus estados mais prováveis; 
	($ii$) \textbf{Fórmulas fechadas para probabilidade de infecção} obtemos,  via método de Newton, fórmulas fechadas para aproximar a  probabilidade de contaminação. As fórmulas são simples e dependem apenas dos parâmetros do sistema; 
	($iii)$ \textbf{Simulação} executamos simulações e verificamos que o comportamento capturado pelo modelo analítico é também observado no  sistema  simulado. Em particular, as simulações levam em conta nós que entram e saem da rede assim como tempos entre eventos gerais (e.g., determinísticos), enquanto que o modelo analítico assume que todos os tempos entre eventos são exponencialmente distribuídos.
	($iv$) \textbf{Caracterização da resposta de sistemas distribuídos} por meio de modelos de filas M/G/K, apresentamos a resposta de sistemas distribuídos para diferentes configurações e regimes de trabalho, inclusive sob ataques de negação de serviço por meio do consumo dos recursos disponíveis.


	\textbf{Organização}
	o restante deste texto está organizado da seguinte forma. O Capítulo~\ref{cap:descricao} descreve o sistema em questão, seguido pela caracterização de usuários reais na Capítulo~\ref{cap:evidencias}, e o modelo na Capítulo~\ref{cap:modelo_epidemico} com algumas fórmulas fechadas. Apresentamos simulações na Capítulo~\ref{cap:simulacao}.   Finalmente, trabalhos relacionados e conclusão vêm nos Capítulos~\ref{cap:trab_relac} e~\ref{cap:conclusao}.