
% ---
% Conclusão
% ---
\chapter{Conclusão}
\label{cap:conclusao}
% ---


Neste trabalho, consideramos a  caracterização do processo de propagação de epidemias frente atacantes estratégicos. Em particular, considerando o modelo analítico previamente proposto em~\cite{rufino2018contaminaccao, rufino2019dilemas},  propusemos um método iterativo para calcular a probabilidade de infecção dos nós.   Comparando os resultados analíticos  com resultados de simulações, observamos que o modelo captura o comportamento do sistema mesmo quando consideramos nós intermitentes, bem como tempos entre eventos que não sejam exponencialmente distribuídos. Acreditamos que os resultados apresentados neste artigo constituam um passo no sentido de estabelecer os fundamentos para melhor modelagem e análise   do processo de propagação de epidemias.  Tal entendimento é crítico para auxiliar na tomada de decisões, por exemplo, sobre vacinar ou reiniciar sistemas conectados à rede, levando em conta os custos de tais contramedidas e os respectivos riscos da não implementação das mesmas.
