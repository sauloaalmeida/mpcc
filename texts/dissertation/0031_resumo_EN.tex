
% resumo em inglês
\begin{resumo}[Abstract]
 \begin{otherlanguage*}{english}
   Every day systems are compromised by malicious code and participate in campaigns to attack computer systems.
Thus, users face a dilemma as to which countermeasures to take: harsh (for example, vaccination), mild (for example, restart and rejuvenation) or no countermeasures. To resolve this dilemma, one option is to take advantage of analytical models. In this work, we present a proposal to characterize the steady state of epidemic models in which the attacker is strategic and has a finite attack capacity. For that, the most probable states of the model are analyzed, indicating their properties; closed formulas are presented that approximate the probability of infection and we contrast the \textit{insights} of the model with simulations. Simulations qualitatively support the observations of the epidemic model and extend the analysis by allowing general distributions.

   \vspace{\onelineskip}
 
   \noindent 
   \textbf{Keywords}: latex. epidemic, network contamination, information security.
 \end{otherlanguage*}
\end{resumo}