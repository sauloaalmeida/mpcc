
% resumo em francês 
\begin{resumo}[Résumé]
 \begin{otherlanguage*}{french}
    Chaque jour, les systèmes sont compromis par un code malveillant et participent à des campagnes pour attaquer les systèmes informatiques.
    Ainsi, les utilisateurs sont confrontés à un dilemme quant aux contre-mesures à prendre: sévères (par exemple, vaccination), douces (par exemple, redémarrage et rajeunissement) ou aucune contre-mesure. Pour résoudre ce dilemme, une option consiste à tirer parti des modèles analytiques. Dans ce travail, nous présentons une proposition pour caractériser l'état stationnaire des modèles épidémiques dans lesquels l'attaquant est stratégique et a une capacité d'attaque finie. Pour cela, les états les plus probables du modèle sont analysés, indiquant leurs propriétés; des formules fermées sont présentées qui approchent la probabilité d'infection et nous contrastons les \emph{insights} du modèle avec des simulations. Les simulations soutiennent qualitativement les observations du modèle épidémique et étendent l'analyse en permettant des distributions générales.
 
   \textbf{Mots-clés}: épidémie, contamination du réseau, sécurité de l'information.
 \end{otherlanguage*}
\end{resumo}